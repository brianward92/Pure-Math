\documentclass[12pt]{article}
\usepackage{amssymb}
\usepackage{amsmath}
\usepackage{natbib}
\usepackage{color}
\usepackage{graphicx}
\usepackage[margin = 0.5in]{geometry}
\usepackage{hyperref}
\usepackage{listings}
\usepackage{float}
\usepackage{rotating}
\usepackage{url}
\usepackage[font=small]{caption}
\usepackage{setspace}
\usepackage{subfigure}
\usepackage[dvipsnames]{xcolor}
\setcounter{MaxMatrixCols}{10}
\setcounter{tocdepth}{1}
\newtheorem{theorem}{Theorem}
\newtheorem{algorithm}{Algorithm} %[theorem]
\newtheorem{corollary}{Corollary}
\newtheorem{definition}{Definition}
\newtheorem{example}{Example}
\newtheorem{exercise}{Exercise}
\newtheorem{lemma}{Lemma}
\newtheorem{proposition}{Proposition}
\newtheorem{remark}{Remark}
\newenvironment{proof}[1][Proof]{\textbf{#1.} }{\ \rule{0.5em}{0.5em}}

%\newtheorem{acknowledgement}[theorem]{Acknowledgement}
%\newtheorem{axiom}[theorem]{Axiom}
%\newtheorem{case}[theorem]{Case}
%\newtheorem{claim}[theorem]{Claim}
%\newtheorem{conclusion}[theorem]{Conclusion}
%\newtheorem{condition}[theorem]{Condition}
%\newtheorem{conjecture}[theorem]{Conjecture}
%\newtheorem{criterion}[theorem]{Criterion}
%\newtheorem{assumption}[theorem]{Assumption}
%\newtheorem{notation}[theorem]{Notation}
%\newtheorem{problem}[theorem]{Problem}
%\newtheorem{solution}[theorem]{Solution}
%\newtheorem{summary}[theorem]{Summary}
	
\def\N{{\mathbb N}}        % positive integers
\def\Q{{\mathbb Q}}        % rationals
\def\Z{{\mathbb Z}}        % integers
\def\R{{\mathbb R}}        % reals
\def\Rn{{\R^{n}}}          % product of n copies of reals
\def\P{{\mathbb P}}        % probability
\def\E{{\mathbb E}}        % expectation
\def\EQ{{\mathbb E}^{\mathbb Q}}        % expectation
\def\1{{\mathbf 1}}        % indicator
\def\F{{\mathcal F}}        % potential measure
\def\G{{\mathcal G}}        % potential measure
\def\ess{\text{ess}}
\def\var{{\mathop{\mathbf Var}}}    %variance
\def\L{{\mathcal L} \,}
\def\Lhat{{\tilde{\mathcal L} \,}}
\def\setZ{{\mathcal Z}}
\def\setA{{\mathcal A}}
\def\setT{{\mathcal T}}
\def\D{{\mathcal D}}
\def\C{{\mathcal C}}
\def\setE{{\mathcal E}}
\def\Vest{{\mathcal V}}
\def\Cvest{{C}}
\def\Cunvest{{\tilde{C}}}

\def\I{{\mathbf I}}
\def\thetahat{{\hat{\theta}}}
\def\taub{{\hat{\tau}}}
\def\xhat{{\hat{x}}}
\def\xbar{{\bar{x}}}
\def\yhat{{\hat{y}}}
\def\x{{\textcolor[rgb]{0.00,0.00,0.00}{\hat{x}}}}
\def\y{{\textcolor[rgb]{0.00,0.00,0.00}{\hat{y}}}}
\def\v{{\hat{v}}}
\def\Xhat{{\hat{X}}}
\def\G{{\tilde{G}}}
\def\H{{\tilde{H}}}
\def\Yhat{{\hat{Y}}}
\def\Vhat{{\hat{V}}}
\def\Mhat{{\hat{M}}}

%\addtolength{\hoffset}{-1.8cm} \addtolength{\voffset}{-2cm}
%\addtolength{\textheight}{4cm} \addtolength{\textwidth}{3.6cm}

\newcommand{\ward}[1]{\textcolor{red}{#1}}

\numberwithin{theorem}{section}
\numberwithin{equation}{section}
\numberwithin{remark}{section}
\numberwithin{definition}{section}
\numberwithin{theorem}{section}
\numberwithin{lemma}{section}
\numberwithin{example}{section}

\begin{document}
\title{Solutions Guide to Differential Geometry}
\author{Brian Ward\thanks{Email: {bmw2150@columbia.edu}. Corresponding author. }} 
\maketitle
\abstract{Solutions to the textbook ``Elementary Differential Geometry", Revised Second Edition by Barrett O'Neill.}

\tableofcontents

\newpage

\section{Calculus on Euclidean Space}
\subsection{Euclidean Space}
\begin{enumerate}
	\item{
		\begin{enumerate}
			\item[(a)]{$fg^2=(x^2y)(y\sin(z))^2=x^2y^3\sin^2(z)$}
			\item[(b)]{$\frac{\partial f}{\partial x}g+\frac{\partial g}{\partial y}f=(2xy)(y\sin(z))+\sin(z)(x^2y)=(2xy^2+x^2y)\sin(z)$}
			\item[(c)]{$fg=(x^2y)(y\sin(z))=x^2y^2\sin(z)\implies\frac{\partial (fg)}{\partial z}=x^2y^2\cos(z)\implies\frac{\partial^2 (fg)}{\partial y \partial z}=2x^2y\cos(z)$}
			\item[(d)]{$\frac{\partial}{\partial y}\sin(f)=\cos(f)\frac{\partial f}{\partial y}=\cos(x^2y)\cdot2xy$}
		\end{enumerate}
	}
	\item{
		\begin{enumerate}
			\item[(a)]{$1^2\cdot1-1^2\cdot1=0$}
			\item[(b)]{$3^2\cdot(-1)-(-1)^2\cdot\frac{1}{2}=-9.5$}
			\item[(c)]{$a^2\cdot1-1^2\cdot(1-a)=a^2+a-1$}
			\item[(d)]{$t^2\cdot t^2-(t^2)^2\cdot t^3=t^4-t^7$}
		\end{enumerate}
	}
	\item{
		\begin{enumerate}
			\item[(a)]{$\sin(xy)+x\cos(xy)y-y\sin(xz)z=\sin(xy)+xy\cos(xy)-yz\sin(xz)$}
			\item[(b)]{$\frac{\partial f}{\partial x}=\cos(g)\frac{\partial g}{\partial x}=\cos(e^h)e^h\frac{\partial h}{\partial x}=\cos(e^{x^2+y^2+z^2})e^{x^2+y^2+z^2}2x$}
		\end{enumerate}
	}
	\item{
		With a slight abuse of notation\footnote{The abuse is that when we say for example $\frac{\partial f}{\partial x}$ we really mean ``partial derivative of $f$ with respect to the first component." (And similarly for $\frac{\partial f}{\partial y}$ and $\frac{\partial f}{\partial z}$).}, the chain rule in this case gives us $\frac{\partial f}{\partial x}=\frac{\partial h}{\partial x}\frac{\partial g_1}{\partial x}+\frac{\partial h}{\partial y}\frac{\partial g_2}{\partial x}+\frac{\partial h}{\partial z}\frac{\partial g_3}{\partial x}$. For the given $h$, we can write this as $\frac{\partial f}{\partial x}=2x\frac{\partial g_1}{\partial x}-z\frac{\partial g_2}{\partial x}-y\frac{\partial g_3}{\partial x}.$ We need only compute the the partial derivatives with respect to $x$ of each coordinate function.
		\begin{enumerate}
			\item[(a)]{$\frac{\partial f}{\partial x}=2x\cdot 1-z\cdot 0-y\cdot 1=2x$}
			\item[(b)]{$\frac{\partial f}{\partial x}=2x\cdot 0-ze^{x+y}-ye^x=-ze^{x+y}-ye^x$}
			\item[(c)]{$\frac{\partial f}{\partial x}=2x\cdot 1-z\cdot (-1)-y\cdot 1 = 2x+z-y$}
		\end{enumerate}
	}
\end{enumerate}


\end{document}